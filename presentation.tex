\documentclass{beamer} 
\usepackage[latin1]{inputenc}
%\usepackage[brazilian]{babel}
\usepackage{amssymb,amsmath} %para usar simbolos matematicos 
\usepackage{palatino} 
%\usepackage{epstopdf}
%\usepackage{textpos}
%\usepackage{beamerthemesplit} 
\setbeamercovered{transparent} % progress�o transparente 
%\setbeamercovered{invisible}
%\usetheme{Warsaw} % um dos temas
\usetheme{Montpellier} % um dos temas
%\usetheme{Berkeley} % um dos temas
%\usetheme{Antibes} % um dos temas 

\title{Calendaring for Wide Area Networks}
\subtitle{
\textit{Paper Presentation}}
\author{Alaelson Jatob�\\
S�rgio Tim�teo}
\institute{Graduate Program in Electrical Engineering\\
Computation \& Electrical Engineering School\\
University of Campinas} 
\date{June, 18th 2015} 
%\date{\today} 

\pgfdeclareimage[height=0.5cm]{logo}{figures/unicamplogo}
\logo{\pgfuseimage{logo}}

%\logo{\includegraphics[height=0.5cm]{figures/unicamplogo}}


\begin{document} 

\frame{
\titlepage
}



%\section[Outline]{}
\frame{
\transdissolve<1>[duration=0.3]
\frametitle{About the Article}
\textbf{Calendaring for Wide Area Networks.}\\
\vspace{\baselineskip}
\textbf{Srikanth Kandula, Ishai Menache, Roy Schwartz, and Spandana Raj Babbula,
\textit{SIGCOMM} 2014.}\\
\vspace{\baselineskip}
Microsoft Research.
}

%Destaca o sum�rio no in�cio de cada sess�o
\AtBeginSection[]
{
\begin{frame}
\frametitle{Outline}
\tableofcontents[currentsection]
\end{frame}
}

%%%%%%%%%%%%%%%%%%%%%%%%%%%%%%%%%%%%%%%%%%%%%%%%%%%%%%%%%%%%%%%%%%%%%%%%%%%%%%%%
\section{Introduction}
\subsection{Motivation}
\transdissolve<1>[duration=0.3]
\frame{
\frametitle{Motivation} 
\begin{itemize} 
\item \textit{Calendaring}:
\vspace{\baselineskip}
\begin{itemize}
	\item <1->Refer to setting aside future resources so that long term objectives are met in spite of short-term demands for resources;
	\vspace{\baselineskip}
	\item <2->Several large cloud companies operate geo-distributed datacenters connected through a WAN;
	\vspace{\baselineskip}
	\item <3-> Order of Terabits per seconds.
	%\only<3>{Order of Terabits per seconds.}
	
\end{itemize}
\end{itemize}
}

%\subsection{Motivation}
\transdissolve<1>[duration=0.3]
\frame{
\frametitle{Motivation} 
The traffic on the inter-datacenter WAN can be roughly characterized as a mix of two types: 
\begin{description} 
\item [highpri] traffic Comprises of instantaneously arriving demands due to customer facing tra?ffic;
	\begin{itemize}
		\item unpredictable;
		\item smaller than fully capacity;
		\item need to be fully met.
	\end{itemize}

\item [General] Large transfers between datacenters that require WAN.
\end{description}
}
 

\subsection{Objectives}

\frame{ 
\frametitle{Objectives}  
\transdissolve<1>[duration=0.3]
	Characteristics of problems:
	\begin{itemize} 
		\item No delays and zero loss rate for high-priority traffic;
		\item When not limited by network capacity, long-running requests are fully met before the deadline;
		\item When demands exceed network capacity, continue to offer guarantees such as maximizing the minimal fraction of transfers that finish before deadline (fairness), or maximizing a specified total utility function.\\
	\end{itemize} 
	\vspace{\baselineskip}
	\textbf{Satisfying all these goals is hard!!! }

} 


\section{TEMPUS}
\subsection{Goal}
\frame{
\frametitle{TEMPUS: Goal}
\transdissolve<1>[duration=0.3]
\begin{itemize}
	\item The goal of \texttt{Tempus} is to accommodate the long-getm requests, while leaving enough capacity for ad-hoc high-priority requests;
	\item Distribute the long-running requests over time and network paths using available information;
	\item \textit{Key aspects:}
	\begin{itemize}
	\item Graph $G = (V,E)$, where $|V| = n$ (nodes) and $|E| = m$ (edges);
	\item Non negative edge capacities $c : E \rightarrow \mathcal{R}^+ $;
	\item Each long-term request $i$, its deadline is the tuple ($\mathnormal{a_i,b_i,d_i,D_i, s_i,t_i,\mathcal{P}_i}$);	
\end{itemize}
\end{itemize}
}


\frame{
\frametitle{TEMPUS: Goal}
\transdissolve<1>[duration=0.3]

Deadline tuple $(\mathnormal{a_i,b_i,d_i,D_i, s_i,t_i,\mathcal{P}_i})$
\begin{table}[h]
	\centering
		\begin{tabular}{*{2}{cl}}
			\hline\hline
    Parameter & Description \\
    \hline
		$\mathnormal{i}$  & $\mathnormal{request}$ \\
    $\mathnormal{a_i}$  & $\mathnormal{aware\ time}$ \\
		$\mathnormal{b_i}$  & $\mathnormal{begin\ time}$ \\
		$\mathnormal{d_i}$  & $\mathnormal{deadline}$ \\
		$\mathnormal{D_i}$  & $\mathnormal{demand}$ \\
		$\mathnormal{s_i}$  & $\mathnormal{source}$\ node \\
		$\mathnormal{t_i}$  & $\mathnormal{target}$\ node \\
		$\mathnormal{\mathcal{P}_i}$ & path\ collection from $\mathnormal{s_i}$ to $\mathnormal{t_i}$\\
		$\mathnormal{t}$  & $\mathnormal{time}$ \\
		$\mathnormal{e}$  & edge\\
		$\mathnormal{c_e}$  & fraction\ of\ capacity\ of\ edge\\
		
   \hline\hline
		\end{tabular}
	\caption{Long-term requests deadline parameters}
	\label{tab:request_param}
\end{table}

}

\frame{ 
\frametitle{TEMPUS: Goal}
\transdissolve<1>[duration=0.3]

The algorithm has an estimate for the request with:
\vspace{\baselineskip}
\begin{itemize}

	\item High priority has a fraction of the capacity $\mathnormal{c_e}$ of edge $\mathnormal{e}$ $$(\chi_{e,t})$$
	\item Long-term has the remain fraction $$(1-\chi_{e,t})$$
	
\end{itemize}
}

%\section{TEMPUS}
\subsection{Offline version}
\frame{ 
\frametitle{TEMPUS: Offline Version}
\transdissolve<1>[duration=0.3]

An optimization problem:
\vspace{\baselineskip}
\begin{itemize}

	\item Amount of flow need to be allocated in time.
	\item Linear inequalities can be formulated to assert the allocation's feasibility: every time and every edge can't exceed the available capacity;
	\item The traffic can only be routed when there is enough time to serve the total demand;
	\item Maxmizing the minimum fraction of request traffic routed before deadline.
\end{itemize}
\vspace{\baselineskip}
\begin{center}
\textbf{The offline problem assumes perfect future knowledge of all requests.} \textbf{It isn't a practical solution.}
\end{center}
}

\subsection{Online version}
\frame{ 
\frametitle{TEMPUS: Online Version}
\transdissolve<1>[duration=0.3]
\begin{itemize}
	\item The offline version is fairness for future requests;
	\item The offline has high computational cost;
	\item To solve it, the online version TEMPUS reduce the time to serve at each request. 
	\item With this, aims to offer large promises early in the lifetime of the request.
	\item All of the previously promised allocation are retained for future high priority traffic;
	\item None of the work that has been done by the previous request (flow allocation) needs to be redone which speeds-up the computations cost.
\end{itemize}

\vspace{\baselineskip}
\begin{center}
\textbf{Using Linear problem with an approach by Young's method} 
\end{center}

}

\frame{ 
\frametitle{}
\begin{center}
	\Huge Obrigado!
\end{center}
}

\end{document}

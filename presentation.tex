\documentclass{beamer} 
\usepackage[latin1]{inputenc}
%\usepackage[brazilian]{babel}
\usepackage[english]{babel}
\usepackage{amssymb,amsmath} %para usar simbolos matematicos 
\usepackage{palatino} 
%\usepackage{epstopdf}
%\usepackage{textpos}
%\usepackage{beamerthemesplit} 
\setbeamercovered{transparent} % progress�o transparente 
%\setbeamercovered{invisible}
%\usetheme{Warsaw} % um dos temas
\usetheme{Montpellier} % um dos temas
%\usetheme{Berkeley} % um dos temas
%\usetheme{Antibes} % um dos temas 

\defbeamertemplate{description item}{align left}{\insertdescriptionitem\hfill}

\title{Calendaring for Wide Area Networks}
\subtitle{
\textit{Paper Presentation}}
\author{Alaelson Jatob�\\
S�rgio Tim�teo}
\institute{Graduate Program in Electrical Engineering\\
Computation \& Electrical Engineering School\\
University of Campinas} 
\date{June, 18th 2015} 
%\date{\today} 

\pgfdeclareimage[height=0.5cm]{logo}{figures/unicamplogo}
\logo{\pgfuseimage{logo}}

%\logo{\includegraphics[height=0.5cm]{figures/unicamplogo}}


\begin{document} 

\frame{
\titlepage
}



%\section[Outline]{}
\frame{
\transdissolve<1>[duration=0.3]
\frametitle{About the Article}
\textbf{Calendaring for Wide Area Networks.}\\
\vspace{\baselineskip}
\textbf{Srikanth Kandula, Ishai Menache, Roy Schwartz, and Spandana Raj Babbula,
\textit{SIGCOMM} 2014.}\\
\vspace{\baselineskip}
Microsoft Research.
}

%Destaca o sum�rio no in�cio de cada sess�o
\AtBeginSection[]
{
\begin{frame}
\frametitle{Outline}
\tableofcontents[currentsection]
\end{frame}
}

%%%%%%%%%%%%%%%%%%%%%%%%%%%%%%%%%%%%%%%%%%%%%%%%%%%%%%%%%%%%%%%%%%%%%%%%%%%%%%%%
\section{Introduction}
\subsection{Motivation}
\transdissolve<1>[duration=0.3]
\frame{
\frametitle{Motivation} 
\begin{itemize} 
\item \textit{Calendaring}:
\vspace{\baselineskip}
\begin{itemize}
	\item <1->Refer to setting aside future resources so that long term objectives are met in spite of short-term demands for resources;
	\vspace{\baselineskip}
	\item <2->Several large cloud companies operate geo-distributed datacenters connected through a WAN;
	\vspace{\baselineskip}
	\item <3-> Order of Terabits per seconds.
	%\only<3>{Order of Terabits per seconds.}
	
\end{itemize}
\end{itemize}
}

%\subsection{Motivation}
\transdissolve<1>[duration=0.3]
\frame{
\frametitle{Motivation} 
The traffic on the inter-datacenter WAN can be roughly characterized as a mix of two types: 
\begin{description} 
\item [highpri] traffic Comprises of instantaneously arriving demands due to customer facing tra?ffic;
	\begin{itemize}
		\item unpredictable;
		\item smaller than fully capacity;
		\item need to be fully met.
	\end{itemize}

\item [General] Large transfers between datacenters that require WAN.
\end{description}
}
 

\subsection{Objectives}

\frame{ 
\frametitle{Objectives}  
\transdissolve<1>[duration=0.3]
	Characteristics of problems:
	\begin{itemize} 
		\item No delays and zero loss rate for high-priority traffic;
		\item When not limited by network capacity, long-running requests are fully met before the deadline;
		\item When demands exceed network capacity, continue to offer guarantees such as maximizing the minimal fraction of transfers that finish before deadline (fairness), or maximizing a specified total utility function.\\
	\end{itemize} 
	\vspace{\baselineskip}
	\textbf{Satisfying all these goals is hard!!! }

} 


\section{TEMPUS}
\subsection{Goal}
\frame{
\frametitle{TEMPUS: Goal}
\transdissolve<1>[duration=0.3]
\begin{itemize}
	\item The goal of \texttt{Tempus} is to accommodate the long-getm requests, while leaving enough capacity for ad-hoc high-priority requests;
	\item Distribute the long-running requests over time and network paths using available information;
	\item \textit{Key aspects:}
	\begin{itemize}
	\item Graph $G = (V,E)$, where $|V| = n$ (nodes) and $|E| = m$ (edges);
	\item Non negative edge capacities $c : E \rightarrow \mathcal{R}^+ $;
	\item Each long-term request $i$, its deadline is the tuple ($\mathnormal{a_i,b_i,d_i,D_i, s_i,t_i,\mathcal{P}_i}$);	
\end{itemize}
\end{itemize}
}


\frame{
\frametitle{TEMPUS: Goal}
\transdissolve<1>[duration=0.3]

%Deadline tuple $(\mathnormal{a_i,b_i,d_i,D_i, s_i,t_i,\mathcal{P}_i})$
\begin{table}[h]
	\centering
		\begin{tabular}{*{2}{cl}}
			\hline\hline
    Parameter & Description \\
    \hline
		$\mathnormal{i}$  & $\mathnormal{request}$ \\
    $\mathnormal{a_i}$  & $\mathnormal{aware\ time}$ \\
		$\mathnormal{b_i}$  & $\mathnormal{begin\ time}$ \\
		$\mathnormal{d_i}$  & $\mathnormal{deadline}$ \\
		$\mathnormal{D_i}$  & $\mathnormal{demand}$ \\
		$\mathnormal{s_i}$  & $\mathnormal{source}$\ node \\
		$\mathnormal{t_i}$  & $\mathnormal{target}$\ node \\
		$\mathnormal{\mathcal{P}_i}$ & path\ collection from $\mathnormal{s_i}$ to $\mathnormal{t_i}$\\
		$\mathnormal{t}$  & $\mathnormal{time}$ \\
		$\mathnormal{e}$  & edge\\
		$\mathnormal{c_e}$  & fraction\ of\ capacity\ of\ edge\\
		
   \hline\hline
		\end{tabular}
	\caption{Long-term requests deadline parameters}
	\label{tab:request_param}
\end{table}

}

\frame{ 
\frametitle{TEMPUS: Goal}
\transdissolve<1>[duration=0.3]

The algorithm has an estimate for the request with:
\vspace{\baselineskip}
\begin{itemize}

	\item High priority has a fraction of the capacity $\mathnormal{c_e}$ of edge $\mathnormal{e}$ $$(\chi_{e,t})$$
	\item Long-term has the remain fraction $$(1-\chi_{e,t})$$
	
\end{itemize}
}

%\section{TEMPUS}
\subsection{Offline version}
\frame{ 
\frametitle{TEMPUS: Offline Version}
\transdissolve<1>[duration=0.3]

An optimization problem:
\vspace{\baselineskip}
\begin{itemize}

	\item Amount of flow need to be allocated in time.
	\item Linear inequalities can be formulated to assert the allocation's feasibility: every time and every edge can't exceed the available capacity;
	\item The traffic can only be routed when there is enough time to serve the total demand;
	\item Maxmizing the minimum fraction of request traffic routed before deadline.
\end{itemize}
\vspace{\baselineskip}
\begin{center}
\textbf{The offline problem assumes perfect future knowledge of all requests.} \textbf{It isn't a practical solution.}
\end{center}
}

\subsection{Online version}
\frame{ 
\frametitle{TEMPUS: Online Version}
\transdissolve<1>[duration=0.3]
\begin{itemize}
	\item The offline version is fairness for future requests;
	\item The offline has high computational cost;
	\item To solve it, the online version TEMPUS reduce the time to serve at each request. 
	\item With this, aims to offer large promises early in the lifetime of the request.
	\item All of the previously promised allocation are retained for future high priority traffic;
	\item None of the work that has been done by the previous request (flow allocation) needs to be redone which speeds-up the computations cost.
\end{itemize}

\vspace{\baselineskip}
\begin{center}
\textbf{Using Linear problem with an approach by Young's method} 
\end{center}

}

\section{Evaluation}
\subsection{Methodology}
\frame{ 
\frametitle{Evaluation: Methodology}
\transdissolve<1>[duration=0.3]
\begin{itemize}
	\item Microsoft's production WAN: 
	\begin{itemize}
		\item 40 Nodes $\rightarrow$ PoPs;
		\item 280 Edges $\rightarrow$ bundles of optical links;
	\end{itemize} 
	\item Other: larger synthetic topologies;
	\end{itemize}

}
	
\frame{ 
\frametitle{Evaluation: Methodology}
\transdissolve<1>[duration=0.3]
\begin{itemize}
	\item Collected netflow data at every ingress into the WAN;
		\begin{itemize}
			\item Couldn't identify witch flow belongs to a single request;
			\item Interview the cluster operator: distribution of sizes, duration between arrival and hard deadlines
		\end{itemize}
	\item 5 minutes averages of traffic sent between PoPs;
	\begin{itemize}
		\item 1 day worth of data overall;
	\end{itemize}
	\item Highpri traffic identified from discussion with cluster operators;
	\begin{itemize}
		\item[-] Recall:
		\item unpredictable amount of demand that arrives in each time step and has to be completely satisfed;
		\item  The remaining traffic portion, which is the bulk of the bytes on the WAN,
are attributed to large requests;
	\end{itemize}
\end{itemize}

\vspace{\baselineskip}
\begin{center}
%\textbf{Using Linear problem with an approach by Young's method} 
\end{center}
}

\frame{ 
\frametitle{Evaluation: Methodology}
\transdissolve<1>[duration=0.3]
	\begin{itemize}
		\item Given a trafic demand between a pair of PoPs,
		\begin{itemize}
		 		\item repeat distributions to generate large transfer requests;
		\end{itemize}
		\item Metrics:

		\setbeamertemplate{description item}[align left]
		\begin{description}[Total utility or $(\delta)$]
			\item[Load factor] a constante factor to scale the matric demand matrix;
			%\item Load factor 
			%\item[MinFract ($\alpha$)] minimum of the fraction of request that is transferred before deadline; Maximum $\alpha$ is good! 
			\item[MinFract ($\alpha$)] minimum of the fraction of request that is transferred before deadline;
%			\item[Requests finishing before deadline] how many requests finish on time. When the network load is low, carrying as much flow as possible will satisfy all (or most) deadlines. However, as the network load increases, finishing more requests on time requires carefully spreading traffic across available paths and time. Also, there is a tension between finishing more requests vs. offering a minimum guarantee to every request.
			\item[Requests finishing before deadline] a binary value;
%			\item[Total utility or $(\delta)$] The total utility from carrying traffic. A non-negative number per timestep to depict the value per unit demand that is satisfied at that time step. A calendaring solution should maximize utility. However, there is a similar tension between maximizing $\alpha$ (the min. share of a request) versus maximizing $\delta$.
			\item[Total utility or $(\delta)$] from carrying traffic;
%				\item[Promise] A guarantee is only useful if provided pre-facto. Whereas $alpha$ above measures the minimum post-facto fraction that each request gets served, with promise, we measure what the calendaring solution can guarantee ahead of time to each request.
			\item[Promise] a guarantee is only useful if provided pre-facto ahead of time to each request.
		\end{description}

\end{itemize}
}


\frame{ 
\frametitle{Evaluation: Methodology}
\transdissolve<1>[duration=0.3]
	\begin{itemize}
		\item Compared schemes:
		\setbeamertemplate{description item}[align left]
		\begin{description}[Tempus]
			\item[Offline Optimal (Gurobi)] Has two Linear Program (LP) solvers: 
				\begin{itemize}
					\item First: Max $\alpha$;
				 	\item Second: Max $\delta$ as function of $\alpha$
				 \end{itemize}
			\item[Tempus] Has three implementations:
				\begin{itemize}
					\item Offline (Uses Young's method instead LP), Online and Parallel;
				\end{itemize}			
			 
			\item[Greedy MaxFlow] An online solution. Doesn't plan into the future. Maximize the total flow can be served in the current time step; 
			\item[Greedy MaxMinFract] Like above, but maximize only $\alpha$ served for all pending requests.
		\end{description}

\end{itemize}
}


\frame{ 
\frametitle{}
\begin{center}
	\Huge Obrigado!
\end{center}
}

\end{document}
